\section{Computational workflow}
Here we give an walk-through of a typical workflow during the course of computing 
trivalent embeddings and extensions thereof, for $\ol{\Rep(U_q(\gg_2))}$. 
The user inputs the relevant data into the variables \verb|q|, \verb|graph|, \verb|AGamma|; 
\verb|q| represents the complex parameter $q$, 
\verb|graph| represents the ordinary integer-entry adjacency matrix for the graph, and 
\verb|AGamma| represents the symbolic adjacency matrix of the graph (see Subsection~\red{subsec:symbols}). 
From here we take the following steps:
\begin{enumerate}
    \item Come up with equations
    \item Numericalize
    \item Simplify
    \item Numerically solve
    \item Quantum guess to obtain an exact solution
\end{enumerate}

\subsection{Finding a Trivalent Embedding}
The equations to solve are derived by embedding the relations of \cite{tricats}[Def 5.22] into the GPA. That is, we assume that the coefficients of the image of the trivalent vertex are
\[
    \alpha = (\alpha_1, \dots, \alpha_n)
\]
and translate the skein-theoretic equations into equations of the coefficients $\alpha_i$ via the embedding. In the case of computing a GPA-embedding of the $\Lambda_1$ planar algebra, this produces a large list of polynomial (almost; see Remark~\ref{rem:real_assumption}) equations of degrees 1, 2, 3, 4, and 5. 

\begin{remark}\label{rem:real_assumption}
    In reality, since many of the equations we get involve the dagger of the trivalent vertex, terms containing $\ol{\alpha_i}$ appear often. However, for \red{really good reasons}, we make the simplifying assumption that the embedding has all real coefficients. While the justification for this might be intuition-based, we end up with solutions; this is all the justification we require in reality. 
\end{remark}

In the case of the graph $\Gamma_{18b}$ of \cite{} Figure 18b, we find that the $2\to1$ hom-space is 54-dimensional. This graph is reproduced here in Figure~\ref{fig:gamma_18b}. This gives us 54 unknowns $\alpha_1,\dots,\alpha_{54}$ which we must solve for. We solve the linear subsystem to immediately eliminate many variables. The remaining ``active'' variables are
\[
\alpha_{14}, \alpha_{15}, \alpha_{16}, \alpha_{25}, \alpha_{33}, \alpha_{34}, \alpha_{35}, \alpha_{40}, \alpha_{41}, \alpha_{45}, \alpha_{52}, \alpha_{53}, \alpha_{54}.
\]
% Inspection of the quadratic equations reveals that a $\alpha_{15}^2 +\alpha_{16}^2$ term appears 
We then use the solver \verb|BB|, a package in \verb|R|, to compute an approximate solution to this system. Inspection of the numerical solution reveals some interesting patterns. If we assume the numerical approximation is high precision, we extract the following relationships:
\[
        \alpha_{15} = \alpha_{35} = \alpha_{53}, \quad \alpha_{16} = \alpha_{34} = \alpha_{54}, \quad \alpha_{40} = -\alpha_{41}, 
\]
After these further substitutions, we are left to find only a small number of unknown real numbers, constrained by a large number of equations.

At this point we turn to a process we refer to as {\bf quantum guessing} to move from the approximate solution to an exact solution. This process is a heuristic based on the observation that the coefficients of such embeddings often exhibit very rigid patterns. That is, they often contain products, sums, and quotients of powers of quantum integers $[n]_q$ depending on the parameter $q$. To make quantum guesses, we do a large overhead computation of finding many terms of the form
\[
\frac{[n_1]_q^{r_1} [n_2]_q^{r_2}}{[n_3]_q^{r_3}},
\]
and sums thereof, for integers $1\leq n_i\leq n$ and half-integers $-r\leq r_i\leq r$. In addition to quantum guessing, the \texttt{Mathematica} function \texttt{RootReduce} also plays a large role in finding exact expressions for numerical approximations to coefficients. For a more detailed explanation of how \texttt{RootReduce} works, see the documentation: \cite{wolfram_rootreduce}. 


\subsection{Finding an Extension}
The process we use to find a conformal embedding starts off similar to the above process, but has some particularities that are worth noting. Some strong assumptions are made, and then proved to be true once they bear the fruit of a solution. 

% \begin{figure}
%     \centering
%     \includegraphics[width=0.5\linewidth]{}
%     \caption{The graph $\Gamma_{18b}$ of \cite{g2_graphs} Figure 18b.}
%     \label{fig:gamma_18b}
% \end{figure}

For the sake of clarifying the whole process, we'll walk through a computation from start to finish. We'll be working with $q = $ with the graph of Figure~\ref{fig:gamma_18b}, finding the idempotent $P_g$. We begin by solving the cap, fork, loop, trace, dagger, and half-braid equations numerically to high precision. The precision of the solution will crystalize relationships between the coefficients of $P_g$. While the purist might be uncomfortable with the slight circularity to come, all is well that ends well; we'll make sure everything is above board later when we verify the correctness of the solution. We now look for patterns in the coefficients that we can leverage. In this case one quickly finds that the coefficients' moduli lie in the set
\[
\left\{\frac{1}{3}, \rho, r\rho, r^2\rho \right\}
\]
where $\rho = \frac{2}{3+\sqrt{21}}$ and $r = \sqrt{\frac{6}{3+\sqrt{21}}}$. Because the underlying equations are polynomial, the coefficients also come in conjugate pairs. Inspecting the arguments (in degrees) one finds the following unique (truncated) arguments $\theta_i$

\begin{tabular}{c||c c c c c c c c c c c}
$i$ & 1 & 2 & 3 & 4 & 5 & 6 & 7 & 8 & 9 & 10 & 11\\ \hline
$\theta_i$ & 9.65 & 31.04 & 40.6 & 50.3 & 60.0 & 88.9 & 91.04 & 120.0 & 139.3 & 148.9 & 170.3 \\
\end{tabular}

Now, since the coefficients are accurate to high precision, one may extract relationships between them. Take for example the relations
\begin{align*}
\theta_9 & = \theta_{11} - \theta_2 \\
\theta_1 & = \theta_{10} - \theta_9 \\
\theta_3 & = \theta_2 + \theta_1 \\
\theta_4 & = \theta_7 - \theta_3
\end{align*}
Also due to the high precision one experiences some tradeoffs. For instance, we may not assume the appearance of $91.04^\circ$ means we can round to $90^\circ$. On the other hand, we may safely assume that $60.0^\circ$ and $120.0^\circ$ really are exactly $\pi/3$ and $2\pi/3$, respectively. However, only following one's mathematical nose to extract such relationships, while technically valid, is not the way forward, as we'll discuss.

In order to more thoroughly arm ourselves with equations, we use our numerically precise approximation of $P_g$ to derive (as yet unproven) relations. For instance, we can numerically derive the relation
\[
right + \zeta_3 left +\zeta_3^2 bottom = 0.
\]

We now circle back and substitute our symbolic solution with indeterminate arguments $\theta_i$ into the equations defining $P_g$, along with any more relations we ``found'' numerically. One should take care to not eliminate $\theta_i$ variables unnecessarily. The complexity of the answer is in some sense distributed across all of the arguments $\theta_i$ that are in play at a given time. Eliminating too many will concentrate complexity in one variable. Since we're already dealing with computationally expensive operations with all the \verb|Root| and \verb|Log| objects, one can easily come to a solution which, while entirely correct, might take the lifetime of the universe to verify.  

% EOS
%%%%%%%%%%%%%%%%%%%%%%%%%%%%%%%%%%%%%%%%%%%%%%%%%%%%%%%%%%%%%%%%%%%%%%%%%%%%%%%%%%%%% 
