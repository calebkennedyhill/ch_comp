\section{Computational Algebraic Number Theory}

% Explain the number fields.

% How to use polynomials.

% \[
%     \xymatrix@R=40pt@C=65pt{
%     xx \ar[r]^{ff} \ar[drr]_{ff} & yy \ar[r]^{ff} \ar[d]^{K} & zz \\
%     & ww \ar[ur]_{ff} & \\
%     }
% \]



Here we explain some of the ways we do computational algebraic number theory (CANT), both explicity and implicitly.


\subsection{CANT setup}
In the course of finding the projection coefficients and the $\Z_n$-like extension structure constants at level 4, 
it was necessary to work over the field 
\[
    K\coloneq \Q(q, a_5, b_5, b_6, b_7, c_1),
\]
where we have
\[
q = e^{2\pi i/48},\quad a_5 = \frac{2}{2},\quad b_5 = \sqrt{\cdots},\quad b_6 = \sqrt{\cdots},\quad b_7 = \sqrt{\cdots}, \quad c_1 = .
\]
While in theory we should be able to accomplish this directly by using Mathematica's built-in support for algebraic numbers with 
\[
    \verb|ToNumberField[{q, a5, b5, b6, b7, c1}]|,
\]
we found this to be too slow\footnote{
    We naively attempted this direct approach and terminated the computation after about three weeks.}.
Instead, something more clever is required.
We exploit the isomorphism $L(\gamma) \cong L[x]/(\min_{\gamma}(x))$.
Practically, this involves combining Mathematica's lighting-fast (for low degree) \verb|AlgebraicNumber| 
operations with its reasonably quick \verb|PolynomialMod| function.

The first useful observation to make is that we have the following field diagram:
\begin{center}
\[
    \xymatrix@R=40pt@C=35pt{
    \F(q) & \F(a_5) & \F(b_5) & \F(b_6) & \F(b_7) & \F(c_1) &  \\
     &  &  \F \coloneq \Q(g) \ar@{-}[ull]^{4} \ar@{-}[ul]^{4} \ar@{-}[u]^{2} \ar@{-}[ur]^{2} \ar@{-}[urr]^{2} \ar@{-}[urrr]^{2}  & &  &  &  \\
     &  &  & \Q \ar@{-}[ul]^{4} &  &  &  \\
    }
\]
\end{center}
where $g = \sqrt{2+\sqrt{3}}$\footnote{
    Note that $\Q(g) = \Q(\sqrt{2}, \sqrt{3})$. 
    We use $g$ as a primitive element to unlock algebraic number functionalities.
    }.
This observation allows us to reduce high-order expressions in $\Q(q, a_5, b_5, b_6, b_7, l)$ using the minimal polynomials for 
$q$, $a_5$, $b_5$, $b_6$, $b_7$, and $c_1$ to lower-order expressions.
We gain additional speed by working over $\F$ instead of $\Q$.
The minimal polynomials for $q$, $a_5$, $b_5$, $b_6$, $b_7$, and $c_1$ over $\F$ are:
\begin{equation*}
    p_{q}(\eta) = \eta^4 - g\eta^2 + 1
\end{equation*}
\begin{equation*}
    p_{a_5}(\alpha) = \alpha^4 +(6g-2g^3)\alpha^2 + (5 -8g -1g^2 +2g^3)
\end{equation*}
\begin{equation*}
    p_{b_5}(\beta_1) = \beta_1^2 + (1 -1g )
\end{equation*}
\begin{equation*}
    p_{b_6}(\beta_2) = \beta_2^2 + (2 g -g^2 ) 
\end{equation*}
\begin{equation*}
    p_{b_7}(\beta_3) = \beta_3^2 - (2 -8g +2g^3)
\end{equation*}
\begin{equation*}
    p_{c_1}(\lambda) = \lambda^2 - (4 +10g +2g^3)
\end{equation*}
where we represent, e.g., the algebraic number $5 -8g -1g^2 +2g^3 \in\F$ by
\[
    \verb|AlgebraicNumber[Root[1-4#^2+#^4&,4], {5,-8,-1,2}]|
\] 


Thus we are able, without knowing its precise value, to deduce that, for example,
\[
   \frac{a_5 q}{(1+q^2)b_5} = a_5 b_5 q (g_1 + g_2 q^2)
\]
where 
\[
    g_1 = -1-g+g^2, \quad\text{and}\quad g_2 = \frac{1}{2}(1 + g + g^2 + g^3)
\]
by reducing modulo the minimal polynomials.

We implement this logic in Mathematica by giving $g$ as 
\[
   \verb|AlgebraicNumber[Root[1-4#^2+#^4&,4], {0, 1, 0, 0}]| 
\]
and reducing the expression $\frac{\alpha \eta}{(1+\eta^2)\beta_1}$ modulo the minimal polynomials;
reducing the polynomials we encoutner modulo quadratic and quartic 
minimal polynomials with \verb|AlgebraicNumber| coefficients is quite fast.




\subsection{CANT for embeddings}
This approach derives efficiency in the two main tasks we have to accomplish: 
finding GPA embeddings, and validating equations among generators.

The process of finding a GPA embedding is a dance back and forth between approximations and exact solutions,
with a heavy theme of making strong assumptions.
Any assumption we make, however, will eventually be verified precisely.
Many times we initially find a numerical approximation to an embedding, and use this to inform our guesses for exact solutions.
A close approximation of a solution gives strong hypotheses for questions of the following type:
\begin{itemize}
    \item Which coordinates are zero?
    \item Which pairs of coordinates are the same?
    \item Which pairs of coordinates are conjugate?
    \item Which coordniates have good immediate guesses? (For example, $-0.4999 + 0.866025 i$ is likely $e^{2\pi i/3}$.)
\end{itemize}


Suppose we have obtained an embedding of $\GG_2(q)$:
\[
    F\left( \skein{/skein_figs/trivalent}{0.075} \right) 
    = \sum_{i=1}^{N} \alpha_i (p_i,q_i) \in \Hom_{\GPA(\Gamma)}(2\to 1).
\]
and we are searching for the image of the second generator for $\DD_4$: 
$F\left( \skein{/skein_figs/Pg}{0.075} \right)$.
In practice we find that for each $i$, $\alpha_i\in \R$.
Furthermore, the set of distinct magnitudes,
\[
    \{ |\alpha_i| \}
\]
is quite small.
At level 4, we have only 9 distinct magnitudes of trivalent coordinates, despite $\Hom_{\GPA(\Gamma)}(2\to 1)$ being 88.
Upon analysis of these nine magnitudes, we find that each lies within the number field generated by one of four of them.
These generators are
\[
    a_5 = \frac{2}{2},\quad b_5 = \sqrt{\cdots},\quad b_6 = \sqrt{\cdots},\quad\text{and}\quad b_7 = \sqrt{\cdots}.
\]
We therefore decide that we will work, in practice, with symbolic variables $\alpha$, $\beta_1$, $\beta_2$, and $\beta_3$,
and reduce any expressions we encounter modulo the minimal polynomials for the generators 
{\it in the variables $\alpha$, $\beta_1$, $\beta_2$, and $\beta_3$}.
Now, in order to express summands of the defining relations of $\DD_4$ such as 
\[
    \skein{/skein_figs/trigon_LHS}{0.1},
\]
we need access to rigidity maps 
\[
    \skein{/skein_figs/cap}{0.075} \quad\text{and}\quad \skein{/skein_figs/cup}{0.075}.
\]
By definition of $\GPA(\Gamma)$, we know the coordinates of these maps.
In order to apply the same CANT techniques, we use symbolic variables for these coordinates: 
$\lambda$ and $\frac{1}{\lambda} = \frac{\lambda}{4 +10g +2g^3}$.
In this way $\lambda$ plays the role of $c_1$.
The final symbol we need is $\eta$, which will play the role of $q$.

Now suppose we find a numerical approximation for the image of the projection in $\GPA(\Gamma)$.
We may arrive at this point by solving a large linear system (i.e., the half-braid relation),
by applying optimization to some nonlinear relations, or some other means.
We can use this approximation to solve, for example, the linear system
\[
    \skein{/skein_figs/trigon_LHS}{0.1} 
    = t_1 \skein{/skein_figs/trivalent}{0.15} + t_2 \skein{/skein_figs/triv_rightUp}{0.1}
\]
to find guesses for the values of $t_1$ and $t_2$.
Once we have our {\it guesses} for $t_1$ and $t_2$, we may turn this into an {\it assumption} by 
imposing the exact, symbolic relationship and returning equations which are linear in the 
coordinates of $F\left( \skein{/skein_figs/Pg}{0.075} \right)$.





\subsection{CANT for equation validation}
Suppose we have an equation in the two generators: trivalent and projection (and cups, caps, and sticks).
With symbolic GPA embeddings of the generators as above, this amounts to a linear combination of vectors,
where the vectors have symbolic polynomial entries and are scaled by symbolic polynomials.
In order to prove these equations are true, we can either specialize the symbols to their numeric values, 
or show that the equations hold purely symbolically.
It turns out the latter approach is superior in the cases we are interested in.

We actually require very few relationships between the symbolic variables to hold 
in order to show that the cyclic extension equations are true.
The following relations, in addition to the defining relations from the minimal 
polynomials, turned out to be sufficient to verify all equations at level 4:
\begin{align*}
\beta_3 & = \beta_1\lambda(\frac{3}{2} -\frac{3}{2}g -\frac{1}{2}g^2 \frac{1}{2}g^3) \\
\beta_1 & = -\alpha^3(\frac{1}{2} +\frac{1}{2}g) -\alpha(1 +\frac{3}{2}g -g^2 -\frac{1}{2}g^3) \\
\beta_3\lambda & == \alpha((-5 -9g +3g^2 +3g^3) +\alpha^2(-2 -5g +g^3)).
\end{align*}

Sufficiency here means that the following process terminates.
Suppose we have an expression $\EE$ which we would like to show is equal to 0.
Impose the known relations on $\EE$: those given by the minimal polynomials in addition 
to those listed above.
Expand the result $\EE_1$ and combine all like terms.
Once again impose the relations on $\EE_1$ to obtain $\EE_2$; combine like terms again.
In the cases we worked with, $\EE_2$ was identically 0 after combining like terms.